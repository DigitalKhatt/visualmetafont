\suraline{سُورَةُ النَّبَإِ}
\bismline{بِسْمِ ٱللَّهِ ٱلرَّحْمَٰنِ ٱلرَّحِيمِ}
عَمَّ يَتَسَآءَلُونَ ۝١ عَنِ ٱلنَّبَإِ ٱلْعَظِيمِ ۝٢ ٱلَّذِي هُمْ فِيهِ مُخْتَلِفُونَ ۝٣
كَلَّا سَيَعْلَمُونَ ۝٤ ثُمَّ كَلَّا سَيَعْلَمُونَ ۝٥ أَلَمْ نَجْعَلِ ٱلْأَرْضَ مِهَٰدࣰا ۝٦
وَٱلْجِبَالَ أَوْتَادࣰا ۝٧ وَخَلَقْنَٰكُمْ أَزْوَٰجࣰا ۝٨ وَجَعَلْنَا نَوْمَكُمْ سُبَاتࣰا ۝٩
وَجَعَلْنَا ٱلَّيْلَ لِبَاسࣰا ۝١٠ وَجَعَلْنَا ٱلنَّهَارَ مَعَاشࣰا ۝١١ وَبَنَيْنَا
فَوْقَكُمْ سَبْعࣰا شِدَادࣰا ۝١٢ وَجَعَلْنَا سِرَاجࣰا وَهَّاجࣰا ۝١٣ وَأَنزَلْنَا مِنَ
ٱلْمُعْصِرَٰتِ مَآءࣰ ثَجَّاجࣰا ۝١٤ لِّنُخْرِجَ بِهِۦ حَبࣰّا وَنَبَاتࣰا ۝١٥ وَجَنَّٰتٍ
أَلْفَافًا ۝١٦ إِنَّ يَوْمَ ٱلْفَصْلِ كَانَ مِيقَٰتࣰا ۝١٧ يَوْمَ يُنفَخُ فِي ٱلصُّورِ
فَتَأْتُونَ أَفْوَاجࣰا ۝١٨ وَفُتِحَتِ ٱلسَّمَآءُ فَكَانَتْ أَبْوَٰبࣰا ۝١٩ وَسُيِّرَتِ
ٱلْجِبَالُ فَكَانَتْ سَرَابًا ۝٢٠ إِنَّ جَهَنَّمَ كَانَتْ مِرْصَادࣰا ۝٢١ لِّلطَّٰغِينَ
مَـَٔابࣰا ۝٢٢ لَّٰبِثِينَ فِيهَآ أَحْقَابࣰا ۝٢٣ لَّا يَذُوقُونَ فِيهَا بَرْدࣰا وَلَا
شَرَابًا ۝٢٤ إِلَّا حَمِيمࣰا وَغَسَّاقࣰا ۝٢٥ جَزَآءࣰ وِفَاقًا ۝٢٦ إِنَّهُمْ كَانُوا۟
لَا يَرْجُونَ حِسَابࣰا ۝٢٧ وَكَذَّبُوا۟ بِـَٔايَٰتِنَا كِذَّابࣰا ۝٢٨ وَكُلَّ شَيْءٍ
أَحْصَيْنَٰهُ كِتَٰبࣰا ۝٢٩ فَذُوقُوا۟ فَلَن نَّزِيدَكُمْ إِلَّا عَذَابًا ۝٣٠
إِنَّ لِلْمُتَّقِينَ مَفَازًا ۝٣١ حَدَآئِقَ وَأَعْنَٰبࣰا ۝٣٢ وَكَوَاعِبَ أَتْرَابࣰا ۝٣٣ وَكَأْسࣰا
دِهَاقࣰا ۝٣٤ لَّا يَسْمَعُونَ فِيهَا لَغْوࣰا وَلَا كِذَّٰبࣰا ۝٣٥ جَزَآءࣰ مِّن رَّبِّكَ عَطَآءً
حِسَابࣰا ۝٣٦ رَّبِّ ٱلسَّمَٰوَٰتِ وَٱلْأَرْضِ وَمَا بَيْنَهُمَا ٱلرَّحْمَٰنِۖ لَا يَمْلِكُونَ
مِنْهُ خِطَابࣰا ۝٣٧ يَوْمَ يَقُومُ ٱلرُّوحُ وَٱلْمَلَٰٓئِكَةُ صَفࣰّاۖ لَّا يَتَكَلَّمُونَ
إِلَّا مَنْ أَذِنَ لَهُ ٱلرَّحْمَٰنُ وَقَالَ صَوَابࣰا ۝٣٨ ذَٰلِكَ ٱلْيَوْمُ ٱلْحَقُّۖ فَمَن
شَآءَ ٱتَّخَذَ إِلَىٰ رَبِّهِۦ مَـَٔابًا ۝٣٩ إِنَّآ أَنذَرْنَٰكُمْ عَذَابࣰا قَرِيبࣰا يَوْمَ يَنظُرُ
ٱلْمَرْءُ مَا قَدَّمَتْ يَدَاهُ وَيَقُولُ ٱلْكَافِرُ يَٰلَيْتَنِي كُنتُ تُرَٰبَۢا ۝٤٠
\suraline{سُورَةُ النَّازِعَاتِ}
\bismline{بِسْمِ ٱللَّهِ ٱلرَّحْمَٰنِ ٱلرَّحِيمِ}
وَٱلنَّٰزِعَٰتِ غَرْقࣰا ۝١ وَٱلنَّٰشِطَٰتِ نَشْطࣰا ۝٢ وَٱلسَّٰبِحَٰتِ سَبْحࣰا ۝٣
فَٱلسَّٰبِقَٰتِ سَبْقࣰا ۝٤ فَٱلْمُدَبِّرَٰتِ أَمْرࣰا ۝٥ يَوْمَ تَرْجُفُ ٱلرَّاجِفَةُ ۝٦
تَتْبَعُهَا ٱلرَّادِفَةُ ۝٧ قُلُوبࣱ يَوْمَئِذࣲ وَاجِفَةٌ ۝٨ أَبْصَٰرُهَا خَٰشِعَةࣱ ۝٩
يَقُولُونَ أَءِنَّا لَمَرْدُودُونَ فِي ٱلْحَافِرَةِ ۝١٠ أَءِذَا كُنَّا عِظَٰمࣰا نَّخِرَةࣰ ۝١١ قَالُوا۟
تِلْكَ إِذࣰا كَرَّةٌ خَاسِرَةࣱ ۝١٢ فَإِنَّمَا هِيَ زَجْرَةࣱ وَٰحِدَةࣱ ۝١٣ فَإِذَا هُم بِٱلسَّاهِرَةِ ۝١٤
هَلْ أَتَىٰكَ حَدِيثُ مُوسَىٰٓ ۝١٥ إِذْ نَادَىٰهُ رَبُّهُۥ بِٱلْوَادِ ٱلْمُقَدَّسِ طُوًى ۝١٦
ٱذْهَبْ إِلَىٰ فِرْعَوْنَ إِنَّهُۥ طَغَىٰ ۝١٧ فَقُلْ هَل لَّكَ إِلَىٰٓ أَن تَزَكَّىٰ ۝١٨
وَأَهْدِيَكَ إِلَىٰ رَبِّكَ فَتَخْشَىٰ ۝١٩ فَأَرَىٰهُ ٱلْأٓيَةَ ٱلْكُبْرَىٰ ۝٢٠
فَكَذَّبَ وَعَصَىٰ ۝٢١ ثُمَّ أَدْبَرَ يَسْعَىٰ ۝٢٢ فَحَشَرَ فَنَادَىٰ ۝٢٣
فَقَالَ أَنَا۠ رَبُّكُمُ ٱلْأَعْلَىٰ ۝٢٤ فَأَخَذَهُ ٱللَّهُ نَكَالَ ٱلْأٓخِرَةِ وَٱلْأُولَىٰٓ ۝٢٥
إِنَّ فِي ذَٰلِكَ لَعِبْرَةࣰ لِّمَن يَخْشَىٰٓ ۝٢٦ ءَأَنتُمْ أَشَدُّ خَلْقًا أَمِ ٱلسَّمَآءُۚ
بَنَىٰهَا ۝٢٧ رَفَعَ سَمْكَهَا فَسَوَّىٰهَا ۝٢٨ وَأَغْطَشَ لَيْلَهَا وَأَخْرَجَ
ضُحَىٰهَا ۝٢٩ وَٱلْأَرْضَ بَعْدَ ذَٰلِكَ دَحَىٰهَآ ۝٣٠ أَخْرَجَ مِنْهَا مَآءَهَا
وَمَرْعَىٰهَا ۝٣١ وَٱلْجِبَالَ أَرْسَىٰهَا ۝٣٢ مَتَٰعࣰا لَّكُمْ وَلِأَنْعَٰمِكُمْ ۝٣٣
فَإِذَا جَآءَتِ ٱلطَّآمَّةُ ٱلْكُبْرَىٰ ۝٣٤ يَوْمَ يَتَذَكَّرُ ٱلْإِنسَٰنُ مَا سَعَىٰ ۝٣٥
وَبُرِّزَتِ ٱلْجَحِيمُ لِمَن يَرَىٰ ۝٣٦ فَأَمَّا مَن طَغَىٰ ۝٣٧ وَءَاثَرَ ٱلْحَيَوٰةَ
ٱلدُّنْيَا ۝٣٨ فَإِنَّ ٱلْجَحِيمَ هِيَ ٱلْمَأْوَىٰ ۝٣٩ وَأَمَّا مَنْ خَافَ مَقَامَ
رَبِّهِۦ وَنَهَى ٱلنَّفْسَ عَنِ ٱلْهَوَىٰ ۝٤٠ فَإِنَّ ٱلْجَنَّةَ هِيَ ٱلْمَأْوَىٰ ۝٤١
يَسْـَٔلُونَكَ عَنِ ٱلسَّاعَةِ أَيَّانَ مُرْسَىٰهَا ۝٤٢ فِيمَ أَنتَ مِن
ذِكْرَىٰهَآ ۝٤٣ إِلَىٰ رَبِّكَ مُنتَهَىٰهَآ ۝٤٤ إِنَّمَآ أَنتَ مُنذِرُ مَن يَخْشَىٰهَا ۝٤٥
كَأَنَّهُمْ يَوْمَ يَرَوْنَهَا لَمْ يَلْبَثُوٓا۟ إِلَّا عَشِيَّةً أَوْ ضُحَىٰهَا ۝٤٦
\suraline{سُورَةُ عَبَسَ}
\bismline{بِسْمِ ٱللَّهِ ٱلرَّحْمَٰنِ ٱلرَّحِيمِ}
عَبَسَ وَتَوَلَّىٰٓ ۝١ أَن جَآءَهُ ٱلْأَعْمَىٰ ۝٢ وَمَا يُدْرِيكَ لَعَلَّهُۥ يَزَّكَّىٰٓ ۝٣ أَوْ يَذَّكَّرُ
فَتَنفَعَهُ ٱلذِّكْرَىٰٓ ۝٤ أَمَّا مَنِ ٱسْتَغْنَىٰ ۝٥ فَأَنتَ لَهُۥ تَصَدَّىٰ ۝٦ وَمَا عَلَيْكَ
أَلَّا يَزَّكَّىٰ ۝٧ وَأَمَّا مَن جَآءَكَ يَسْعَىٰ ۝٨ وَهُوَ يَخْشَىٰ ۝٩ فَأَنتَ عَنْهُ تَلَهَّىٰ ۝١٠
كَلَّآ إِنَّهَا تَذْكِرَةࣱ ۝١١ فَمَن شَآءَ ذَكَرَهُۥ ۝١٢ فِي صُحُفࣲ مُّكَرَّمَةࣲ ۝١٣ مَّرْفُوعَةࣲ
مُّطَهَّرَةِۭ ۝١٤ بِأَيْدِي سَفَرَةࣲ ۝١٥ كِرَامِۭ بَرَرَةࣲ ۝١٦ قُتِلَ ٱلْإِنسَٰنُ مَآ أَكْفَرَهُۥ ۝١٧ مِنْ
أَيِّ شَيْءٍ خَلَقَهُۥ ۝١٨ مِن نُّطْفَةٍ خَلَقَهُۥ فَقَدَّرَهُۥ ۝١٩ ثُمَّ ٱلسَّبِيلَ يَسَّرَهُۥ ۝٢٠
ثُمَّ أَمَاتَهُۥ فَأَقْبَرَهُۥ ۝٢١ ثُمَّ إِذَا شَآءَ أَنشَرَهُۥ ۝٢٢ كَلَّا لَمَّا يَقْضِ مَآ أَمَرَهُۥ ۝٢٣
فَلْيَنظُرِ ٱلْإِنسَٰنُ إِلَىٰ طَعَامِهِۦٓ ۝٢٤ أَنَّا صَبَبْنَا ٱلْمَآءَ صَبࣰّا ۝٢٥ ثُمَّ شَقَقْنَا
ٱلْأَرْضَ شَقࣰّا ۝٢٦ فَأَنۢبَتْنَا فِيهَا حَبࣰّا ۝٢٧ وَعِنَبࣰا وَقَضْبࣰا ۝٢٨ وَزَيْتُونࣰا وَنَخْلࣰا ۝٢٩
وَحَدَآئِقَ غُلْبࣰا ۝٣٠ وَفَٰكِهَةࣰ وَأَبࣰّا ۝٣١ مَّتَٰعࣰا لَّكُمْ وَلِأَنْعَٰمِكُمْ ۝٣٢ فَإِذَا جَآءَتِ
ٱلصَّآخَّةُ ۝٣٣ يَوْمَ يَفِرُّ ٱلْمَرْءُ مِنْ أَخِيهِ ۝٣٤ وَأُمِّهِۦ وَأَبِيهِ ۝٣٥ وَصَٰحِبَتِهِۦ
وَبَنِيهِ ۝٣٦ لِكُلِّ ٱمْرِئࣲ مِّنْهُمْ يَوْمَئِذࣲ شَأْنࣱ يُغْنِيهِ ۝٣٧ وُجُوهࣱ يَوْمَئِذࣲ
مُّسْفِرَةࣱ ۝٣٨ ضَاحِكَةࣱ مُّسْتَبْشِرَةࣱ ۝٣٩ وَوُجُوهࣱ يَوْمَئِذٍ عَلَيْهَا غَبَرَةࣱ ۝٤٠
تَرْهَقُهَا قَتَرَةٌ ۝٤١ أُو۟لَٰٓئِكَ هُمُ ٱلْكَفَرَةُ ٱلْفَجَرَةُ ۝٤٢
\suraline{سُورَةُ التَّكْوِيرِ}
\bismline{بِسْمِ ٱللَّهِ ٱلرَّحْمَٰنِ ٱلرَّحِيمِ}
إِذَا ٱلشَّمْسُ كُوِّرَتْ ۝١ وَإِذَا ٱلنُّجُومُ ٱنكَدَرَتْ ۝٢ وَإِذَا ٱلْجِبَالُ
سُيِّرَتْ ۝٣ وَإِذَا ٱلْعِشَارُ عُطِّلَتْ ۝٤ وَإِذَا ٱلْوُحُوشُ حُشِرَتْ ۝٥
وَإِذَا ٱلْبِحَارُ سُجِّرَتْ ۝٦ وَإِذَا ٱلنُّفُوسُ زُوِّجَتْ ۝٧ وَإِذَا ٱلْمَوْءُۥدَةُ
سُئِلَتْ ۝٨ بِأَيِّ ذَنۢبࣲ قُتِلَتْ ۝٩ وَإِذَا ٱلصُّحُفُ نُشِرَتْ ۝١٠
وَإِذَا ٱلسَّمَآءُ كُشِطَتْ ۝١١ وَإِذَا ٱلْجَحِيمُ سُعِّرَتْ ۝١٢ وَإِذَا ٱلْجَنَّةُ
أُزْلِفَتْ ۝١٣ عَلِمَتْ نَفْسࣱ مَّآ أَحْضَرَتْ ۝١٤ فَلَآ أُقْسِمُ بِٱلْخُنَّسِ ۝١٥
ٱلْجَوَارِ ٱلْكُنَّسِ ۝١٦ وَٱلَّيْلِ إِذَا عَسْعَسَ ۝١٧ وَٱلصُّبْحِ إِذَا تَنَفَّسَ ۝١٨
إِنَّهُۥ لَقَوْلُ رَسُولࣲ كَرِيمࣲ ۝١٩ ذِي قُوَّةٍ عِندَ ذِي ٱلْعَرْشِ مَكِينࣲ ۝٢٠ مُّطَاعࣲ ثَمَّ
أَمِينࣲ ۝٢١ وَمَا صَاحِبُكُم بِمَجْنُونࣲ ۝٢٢ وَلَقَدْ رَءَاهُ بِٱلْأُفُقِ ٱلْمُبِينِ ۝٢٣
وَمَا هُوَ عَلَى ٱلْغَيْبِ بِضَنِينࣲ ۝٢٤ وَمَا هُوَ بِقَوْلِ شَيْطَٰنࣲ رَّجِيمࣲ ۝٢٥
فَأَيْنَ تَذْهَبُونَ ۝٢٦ إِنْ هُوَ إِلَّا ذِكْرࣱ لِّلْعَٰلَمِينَ ۝٢٧ لِمَن شَآءَ مِنكُمْ أَن
يَسْتَقِيمَ ۝٢٨ وَمَا تَشَآءُونَ إِلَّآ أَن يَشَآءَ ٱللَّهُ رَبُّ ٱلْعَٰلَمِينَ ۝٢٩
\suraline{سُورَةُ الانفِطَارِ}
\bismline{بِسْمِ ٱللَّهِ ٱلرَّحْمَٰنِ ٱلرَّحِيمِ}
إِذَا ٱلسَّمَآءُ ٱنفَطَرَتْ ۝١ وَإِذَا ٱلْكَوَاكِبُ ٱنتَثَرَتْ ۝٢ وَإِذَا ٱلْبِحَارُ
فُجِّرَتْ ۝٣ وَإِذَا ٱلْقُبُورُ بُعْثِرَتْ ۝٤ عَلِمَتْ نَفْسࣱ مَّا قَدَّمَتْ
وَأَخَّرَتْ ۝٥ يَٰٓأَيُّهَا ٱلْإِنسَٰنُ مَا غَرَّكَ بِرَبِّكَ ٱلْكَرِيمِ ۝٦ ٱلَّذِي
خَلَقَكَ فَسَوَّىٰكَ فَعَدَلَكَ ۝٧ فِيٓ أَيِّ صُورَةࣲ مَّا شَآءَ رَكَّبَكَ ۝٨
كَلَّا بَلْ تُكَذِّبُونَ بِٱلدِّينِ ۝٩ وَإِنَّ عَلَيْكُمْ لَحَٰفِظِينَ ۝١٠ كِرَامࣰا
كَٰتِبِينَ ۝١١ يَعْلَمُونَ مَا تَفْعَلُونَ ۝١٢ إِنَّ ٱلْأَبْرَارَ لَفِي نَعِيمࣲ ۝١٣ وَإِنَّ
ٱلْفُجَّارَ لَفِي جَحِيمࣲ ۝١٤ يَصْلَوْنَهَا يَوْمَ ٱلدِّينِ ۝١٥ وَمَا هُمْ عَنْهَا
بِغَآئِبِينَ ۝١٦ وَمَآ أَدْرَىٰكَ مَا يَوْمُ ٱلدِّينِ ۝١٧ ثُمَّ مَآ أَدْرَىٰكَ مَا يَوْمُ
ٱلدِّينِ ۝١٨ يَوْمَ لَا تَمْلِكُ نَفْسࣱ لِّنَفْسࣲ شَيْـࣰٔاۖ وَٱلْأَمْرُ يَوْمَئِذࣲ لِّلَّهِ ۝١٩
\suraline{سُورَةُ المُطَفِّفِينَ}
\bismline{بِسْمِ ٱللَّهِ ٱلرَّحْمَٰنِ ٱلرَّحِيمِ}
وَيْلࣱ لِّلْمُطَفِّفِينَ ۝١ ٱلَّذِينَ إِذَا ٱكْتَالُوا۟ عَلَى ٱلنَّاسِ يَسْتَوْفُونَ ۝٢
وَإِذَا كَالُوهُمْ أَو وَّزَنُوهُمْ يُخْسِرُونَ ۝٣ أَلَا يَظُنُّ أُو۟لَٰٓئِكَ أَنَّهُم مَّبْعُوثُونَ ۝٤
لِيَوْمٍ عَظِيمࣲ ۝٥ يَوْمَ يَقُومُ ٱلنَّاسُ لِرَبِّ ٱلْعَٰلَمِينَ ۝٦ كَلَّآ إِنَّ كِتَٰبَ
ٱلْفُجَّارِ لَفِي سِجِّينࣲ ۝٧ وَمَآ أَدْرَىٰكَ مَا سِجِّينࣱ ۝٨ كِتَٰبࣱ مَّرْقُومࣱ ۝٩
وَيْلࣱ يَوْمَئِذࣲ لِّلْمُكَذِّبِينَ ۝١٠ ٱلَّذِينَ يُكَذِّبُونَ بِيَوْمِ ٱلدِّينِ ۝١١ وَمَا يُكَذِّبُ
بِهِۦٓ إِلَّا كُلُّ مُعْتَدٍ أَثِيمٍ ۝١٢ إِذَا تُتْلَىٰ عَلَيْهِ ءَايَٰتُنَا قَالَ أَسَٰطِيرُ ٱلْأَوَّلِينَ ۝١٣
كَلَّاۖ بَلْ͏ۜ رَانَ عَلَىٰ قُلُوبِهِم مَّا كَانُوا۟ يَكْسِبُونَ ۝١٤ كَلَّآ إِنَّهُمْ عَن رَّبِّهِمْ
يَوْمَئِذࣲ لَّمَحْجُوبُونَ ۝١٥ ثُمَّ إِنَّهُمْ لَصَالُوا۟ ٱلْجَحِيمِ ۝١٦ ثُمَّ يُقَالُ هَٰذَا
ٱلَّذِي كُنتُم بِهِۦ تُكَذِّبُونَ ۝١٧ كَلَّآ إِنَّ كِتَٰبَ ٱلْأَبْرَارِ لَفِي عِلِّيِّينَ ۝١٨
وَمَآ أَدْرَىٰكَ مَا عِلِّيُّونَ ۝١٩ كِتَٰبࣱ مَّرْقُومࣱ ۝٢٠ يَشْهَدُهُ ٱلْمُقَرَّبُونَ ۝٢١
إِنَّ ٱلْأَبْرَارَ لَفِي نَعِيمٍ ۝٢٢ عَلَى ٱلْأَرَآئِكِ يَنظُرُونَ ۝٢٣ تَعْرِفُ فِي
وُجُوهِهِمْ نَضْرَةَ ٱلنَّعِيمِ ۝٢٤ يُسْقَوْنَ مِن رَّحِيقࣲ مَّخْتُومٍ ۝٢٥ خِتَٰمُهُۥ
مِسْكࣱۚ وَفِي ذَٰلِكَ فَلْيَتَنَافَسِ ٱلْمُتَنَٰفِسُونَ ۝٢٦ وَمِزَاجُهُۥ مِن
تَسْنِيمٍ ۝٢٧ عَيْنࣰا يَشْرَبُ بِهَا ٱلْمُقَرَّبُونَ ۝٢٨ إِنَّ ٱلَّذِينَ أَجْرَمُوا۟ كَانُوا۟
مِنَ ٱلَّذِينَ ءَامَنُوا۟ يَضْحَكُونَ ۝٢٩ وَإِذَا مَرُّوا۟ بِهِمْ يَتَغَامَزُونَ ۝٣٠
وَإِذَا ٱنقَلَبُوٓا۟ إِلَىٰٓ أَهْلِهِمُ ٱنقَلَبُوا۟ فَكِهِينَ ۝٣١ وَإِذَا رَأَوْهُمْ قَالُوٓا۟
إِنَّ هَٰٓؤُلَآءِ لَضَآلُّونَ ۝٣٢ وَمَآ أُرْسِلُوا۟ عَلَيْهِمْ حَٰفِظِينَ ۝٣٣
فَٱلْيَوْمَ ٱلَّذِينَ ءَامَنُوا۟ مِنَ ٱلْكُفَّارِ يَضْحَكُونَ ۝٣٤ عَلَى
ٱلْأَرَآئِكِ يَنظُرُونَ ۝٣٥ هَلْ ثُوِّبَ ٱلْكُفَّارُ مَا كَانُوا۟ يَفْعَلُونَ ۝٣٦
\suraline{سُورَةُ الانشِقَاقِ}
\bismline{بِسْمِ ٱللَّهِ ٱلرَّحْمَٰنِ ٱلرَّحِيمِ}
إِذَا ٱلسَّمَآءُ ٱنشَقَّتْ ۝١ وَأَذِنَتْ لِرَبِّهَا وَحُقَّتْ ۝٢ وَإِذَا ٱلْأَرْضُ مُدَّتْ ۝٣
وَأَلْقَتْ مَا فِيهَا وَتَخَلَّتْ ۝٤ وَأَذِنَتْ لِرَبِّهَا وَحُقَّتْ ۝٥ يَٰٓأَيُّهَا
ٱلْإِنسَٰنُ إِنَّكَ كَادِحٌ إِلَىٰ رَبِّكَ كَدْحࣰا فَمُلَٰقِيهِ ۝٦ فَأَمَّا مَنْ أُوتِيَ
كِتَٰبَهُۥ بِيَمِينِهِۦ ۝٧ فَسَوْفَ يُحَاسَبُ حِسَابࣰا يَسِيرࣰا ۝٨ وَيَنقَلِبُ
إِلَىٰٓ أَهْلِهِۦ مَسْرُورࣰا ۝٩ وَأَمَّا مَنْ أُوتِيَ كِتَٰبَهُۥ وَرَآءَ ظَهْرِهِۦ ۝١٠ فَسَوْفَ
يَدْعُوا۟ ثُبُورࣰا ۝١١ وَيَصْلَىٰ سَعِيرًا ۝١٢ إِنَّهُۥ كَانَ فِيٓ أَهْلِهِۦ مَسْرُورًا ۝١٣
إِنَّهُۥ ظَنَّ أَن لَّن يَحُورَ ۝١٤ بَلَىٰٓۚ إِنَّ رَبَّهُۥ كَانَ بِهِۦ بَصِيرࣰا ۝١٥ فَلَآ أُقْسِمُ
بِٱلشَّفَقِ ۝١٦ وَٱلَّيْلِ وَمَا وَسَقَ ۝١٧ وَٱلْقَمَرِ إِذَا ٱتَّسَقَ ۝١٨
لَتَرْكَبُنَّ طَبَقًا عَن طَبَقࣲ ۝١٩ فَمَا لَهُمْ لَا يُؤْمِنُونَ ۝٢٠ وَإِذَا قُرِئَ
عَلَيْهِمُ ٱلْقُرْءَانُ لَا \sajdabar{يَسْجُدُونَ۩} ۝٢١ بَلِ ٱلَّذِينَ كَفَرُوا۟ يُكَذِّبُونَ ۝٢٢
وَٱللَّهُ أَعْلَمُ بِمَا يُوعُونَ ۝٢٣ فَبَشِّرْهُم بِعَذَابٍ أَلِيمٍ ۝٢٤
إِلَّا ٱلَّذِينَ ءَامَنُوا۟ وَعَمِلُوا۟ ٱلصَّٰلِحَٰتِ لَهُمْ أَجْرٌ غَيْرُ مَمْنُونِۭ ۝٢٥
\suraline{سُورَةُ البُرُوجِ}
\bismline{بِسْمِ ٱللَّهِ ٱلرَّحْمَٰنِ ٱلرَّحِيمِ}
وَٱلسَّمَآءِ ذَاتِ ٱلْبُرُوجِ ۝١ وَٱلْيَوْمِ ٱلْمَوْعُودِ ۝٢ وَشَاهِدࣲ وَمَشْهُودࣲ ۝٣
قُتِلَ أَصْحَٰبُ ٱلْأُخْدُودِ ۝٤ ٱلنَّارِ ذَاتِ ٱلْوَقُودِ ۝٥ إِذْ هُمْ عَلَيْهَا
قُعُودࣱ ۝٦ وَهُمْ عَلَىٰ مَا يَفْعَلُونَ بِٱلْمُؤْمِنِينَ شُهُودࣱ ۝٧ وَمَا نَقَمُوا۟
مِنْهُمْ إِلَّآ أَن يُؤْمِنُوا۟ بِٱللَّهِ ٱلْعَزِيزِ ٱلْحَمِيدِ ۝٨ ٱلَّذِي لَهُۥ مُلْكُ
ٱلسَّمَٰوَٰتِ وَٱلْأَرْضِۚ وَٱللَّهُ عَلَىٰ كُلِّ شَيْءࣲ شَهِيدٌ ۝٩ إِنَّ ٱلَّذِينَ
فَتَنُوا۟ ٱلْمُؤْمِنِينَ وَٱلْمُؤْمِنَٰتِ ثُمَّ لَمْ يَتُوبُوا۟ فَلَهُمْ عَذَابُ جَهَنَّمَ وَلَهُمْ
عَذَابُ ٱلْحَرِيقِ ۝١٠ إِنَّ ٱلَّذِينَ ءَامَنُوا۟ وَعَمِلُوا۟ ٱلصَّٰلِحَٰتِ لَهُمْ
جَنَّٰتࣱ تَجْرِي مِن تَحْتِهَا ٱلْأَنْهَٰرُۚ ذَٰلِكَ ٱلْفَوْزُ ٱلْكَبِيرُ ۝١١ إِنَّ بَطْشَ
رَبِّكَ لَشَدِيدٌ ۝١٢ إِنَّهُۥ هُوَ يُبْدِئُ وَيُعِيدُ ۝١٣ وَهُوَ ٱلْغَفُورُ ٱلْوَدُودُ ۝١٤
ذُو ٱلْعَرْشِ ٱلْمَجِيدُ ۝١٥ فَعَّالࣱ لِّمَا يُرِيدُ ۝١٦ هَلْ أَتَىٰكَ حَدِيثُ
ٱلْجُنُودِ ۝١٧ فِرْعَوْنَ وَثَمُودَ ۝١٨ بَلِ ٱلَّذِينَ كَفَرُوا۟ فِي تَكْذِيبࣲ ۝١٩ وَٱللَّهُ
مِن وَرَآئِهِم مُّحِيطُۢ ۝٢٠ بَلْ هُوَ قُرْءَانࣱ مَّجِيدࣱ ۝٢١ فِي لَوْحࣲ مَّحْفُوظِۭ ۝٢٢
\suraline{سُورَةُ الطَّارِقِ}
\bismline{بِسْمِ ٱللَّهِ ٱلرَّحْمَٰنِ ٱلرَّحِيمِ}
وَٱلسَّمَآءِ وَٱلطَّارِقِ ۝١ وَمَآ أَدْرَىٰكَ مَا ٱلطَّارِقُ ۝٢ ٱلنَّجْمُ ٱلثَّاقِبُ ۝٣
إِن كُلُّ نَفْسࣲ لَّمَّا عَلَيْهَا حَافِظࣱ ۝٤ فَلْيَنظُرِ ٱلْإِنسَٰنُ مِمَّ خُلِقَ ۝٥
خُلِقَ مِن مَّآءࣲ دَافِقࣲ ۝٦ يَخْرُجُ مِنۢ بَيْنِ ٱلصُّلْبِ وَٱلتَّرَآئِبِ ۝٧ إِنَّهُۥ
عَلَىٰ رَجْعِهِۦ لَقَادِرࣱ ۝٨ يَوْمَ تُبْلَى ٱلسَّرَآئِرُ ۝٩ فَمَا لَهُۥ مِن قُوَّةࣲ وَلَا
نَاصِرࣲ ۝١٠ وَٱلسَّمَآءِ ذَاتِ ٱلرَّجْعِ ۝١١ وَٱلْأَرْضِ ذَاتِ ٱلصَّدْعِ ۝١٢
إِنَّهُۥ لَقَوْلࣱ فَصْلࣱ ۝١٣ وَمَا هُوَ بِٱلْهَزْلِ ۝١٤ إِنَّهُمْ يَكِيدُونَ كَيْدࣰا ۝١٥
وَأَكِيدُ كَيْدࣰا ۝١٦ فَمَهِّلِ ٱلْكَٰفِرِينَ أَمْهِلْهُمْ رُوَيْدَۢا ۝١٧
\suraline{سُورَةُ الأَعْلَىٰ}
\bismline{بِسْمِ ٱللَّهِ ٱلرَّحْمَٰنِ ٱلرَّحِيمِ}
سَبِّحِ ٱسْمَ رَبِّكَ ٱلْأَعْلَى ۝١ ٱلَّذِي خَلَقَ فَسَوَّىٰ ۝٢ وَٱلَّذِي قَدَّرَ
فَهَدَىٰ ۝٣ وَٱلَّذِيٓ أَخْرَجَ ٱلْمَرْعَىٰ ۝٤ فَجَعَلَهُۥ غُثَآءً أَحْوَىٰ ۝٥
سَنُقْرِئُكَ فَلَا تَنسَىٰٓ ۝٦ إِلَّا مَا شَآءَ ٱللَّهُۚ إِنَّهُۥ يَعْلَمُ ٱلْجَهْرَ وَمَا يَخْفَىٰ ۝٧
وَنُيَسِّرُكَ لِلْيُسْرَىٰ ۝٨ فَذَكِّرْ إِن نَّفَعَتِ ٱلذِّكْرَىٰ ۝٩ سَيَذَّكَّرُ مَن يَخْشَىٰ ۝١٠
وَيَتَجَنَّبُهَا ٱلْأَشْقَى ۝١١ ٱلَّذِي يَصْلَى ٱلنَّارَ ٱلْكُبْرَىٰ ۝١٢ ثُمَّ لَا يَمُوتُ
فِيهَا وَلَا يَحْيَىٰ ۝١٣ قَدْ أَفْلَحَ مَن تَزَكَّىٰ ۝١٤ وَذَكَرَ ٱسْمَ رَبِّهِۦ فَصَلَّىٰ ۝١٥
بَلْ تُؤْثِرُونَ ٱلْحَيَوٰةَ ٱلدُّنْيَا ۝١٦ وَٱلْأٓخِرَةُ خَيْرࣱ وَأَبْقَىٰٓ ۝١٧ إِنَّ
هَٰذَا لَفِي ٱلصُّحُفِ ٱلْأُولَىٰ ۝١٨ صُحُفِ إِبْرَٰهِيمَ وَمُوسَىٰ ۝١٩
\suraline{سُورَةُ الغَاشِيَةِ}
\bismline{بِسْمِ ٱللَّهِ ٱلرَّحْمَٰنِ ٱلرَّحِيمِ}
هَلْ أَتَىٰكَ حَدِيثُ ٱلْغَٰشِيَةِ ۝١ وُجُوهࣱ يَوْمَئِذٍ خَٰشِعَةٌ ۝٢ عَامِلَةࣱ
نَّاصِبَةࣱ ۝٣ تَصْلَىٰ نَارًا حَامِيَةࣰ ۝٤ تُسْقَىٰ مِنْ عَيْنٍ ءَانِيَةࣲ ۝٥ لَّيْسَ
لَهُمْ طَعَامٌ إِلَّا مِن ضَرِيعࣲ ۝٦ لَّا يُسْمِنُ وَلَا يُغْنِي مِن جُوعࣲ ۝٧ وُجُوهࣱ
يَوْمَئِذࣲ نَّاعِمَةࣱ ۝٨ لِّسَعْيِهَا رَاضِيَةࣱ ۝٩ فِي جَنَّةٍ عَالِيَةࣲ ۝١٠ لَّا تَسْمَعُ
فِيهَا لَٰغِيَةࣰ ۝١١ فِيهَا عَيْنࣱ جَارِيَةࣱ ۝١٢ فِيهَا سُرُرࣱ مَّرْفُوعَةࣱ ۝١٣ وَأَكْوَابࣱ
مَّوْضُوعَةࣱ ۝١٤ وَنَمَارِقُ مَصْفُوفَةࣱ ۝١٥ وَزَرَابِيُّ مَبْثُوثَةٌ ۝١٦ أَفَلَا يَنظُرُونَ
إِلَى ٱلْإِبِلِ كَيْفَ خُلِقَتْ ۝١٧ وَإِلَى ٱلسَّمَآءِ كَيْفَ رُفِعَتْ ۝١٨ وَإِلَى
ٱلْجِبَالِ كَيْفَ نُصِبَتْ ۝١٩ وَإِلَى ٱلْأَرْضِ كَيْفَ سُطِحَتْ ۝٢٠
فَذَكِّرْ إِنَّمَآ أَنتَ مُذَكِّرࣱ ۝٢١ لَّسْتَ عَلَيْهِم بِمُصَيْطِرٍ ۝٢٢
إِلَّا مَن تَوَلَّىٰ وَكَفَرَ ۝٢٣ فَيُعَذِّبُهُ ٱللَّهُ ٱلْعَذَابَ ٱلْأَكْبَرَ ۝٢٤
إِنَّ إِلَيْنَآ إِيَابَهُمْ ۝٢٥ ثُمَّ إِنَّ عَلَيْنَا حِسَابَهُم ۝٢٦
\suraline{سُورَةُ الفَجْرِ}
\bismline{بِسْمِ ٱللَّهِ ٱلرَّحْمَٰنِ ٱلرَّحِيمِ}
وَٱلْفَجْرِ ۝١ وَلَيَالٍ عَشْرࣲ ۝٢ وَٱلشَّفْعِ وَٱلْوَتْرِ ۝٣ وَٱلَّيْلِ إِذَا يَسْرِ ۝٤
هَلْ فِي ذَٰلِكَ قَسَمࣱ لِّذِي حِجْرٍ ۝٥ أَلَمْ تَرَ كَيْفَ فَعَلَ رَبُّكَ بِعَادٍ ۝٦
إِرَمَ ذَاتِ ٱلْعِمَادِ ۝٧ ٱلَّتِي لَمْ يُخْلَقْ مِثْلُهَا فِي ٱلْبِلَٰدِ ۝٨ وَثَمُودَ ٱلَّذِينَ
جَابُوا۟ ٱلصَّخْرَ بِٱلْوَادِ ۝٩ وَفِرْعَوْنَ ذِي ٱلْأَوْتَادِ ۝١٠ ٱلَّذِينَ طَغَوْا۟ فِي
ٱلْبِلَٰدِ ۝١١ فَأَكْثَرُوا۟ فِيهَا ٱلْفَسَادَ ۝١٢ فَصَبَّ عَلَيْهِمْ رَبُّكَ سَوْطَ
عَذَابٍ ۝١٣ إِنَّ رَبَّكَ لَبِٱلْمِرْصَادِ ۝١٤ فَأَمَّا ٱلْإِنسَٰنُ إِذَا مَا ٱبْتَلَىٰهُ
رَبُّهُۥ فَأَكْرَمَهُۥ وَنَعَّمَهُۥ فَيَقُولُ رَبِّيٓ أَكْرَمَنِ ۝١٥ وَأَمَّآ إِذَا مَا ٱبْتَلَىٰهُ
فَقَدَرَ عَلَيْهِ رِزْقَهُۥ فَيَقُولُ رَبِّيٓ أَهَٰنَنِ ۝١٦ كَلَّاۖ بَل لَّا تُكْرِمُونَ
ٱلْيَتِيمَ ۝١٧ وَلَا تَحَٰٓضُّونَ عَلَىٰ طَعَامِ ٱلْمِسْكِينِ ۝١٨ وَتَأْكُلُونَ
ٱلتُّرَاثَ أَكْلࣰا لَّمࣰّا ۝١٩ وَتُحِبُّونَ ٱلْمَالَ حُبࣰّا جَمࣰّا ۝٢٠ كَلَّآۖ إِذَا
دُكَّتِ ٱلْأَرْضُ دَكࣰّا دَكࣰّا ۝٢١ وَجَآءَ رَبُّكَ وَٱلْمَلَكُ صَفࣰّا صَفࣰّا ۝٢٢
وَجِا۟يٓءَ يَوْمَئِذِۭ بِجَهَنَّمَۚ يَوْمَئِذࣲ يَتَذَكَّرُ ٱلْإِنسَٰنُ وَأَنَّىٰ
لَهُ ٱلذِّكْرَىٰ ۝٢٣ يَقُولُ يَٰلَيْتَنِي قَدَّمْتُ لِحَيَاتِي ۝٢٤ فَيَوْمَئِذࣲ
لَّا يُعَذِّبُ عَذَابَهُۥٓ أَحَدࣱ ۝٢٥ وَلَا يُوثِقُ وَثَاقَهُۥٓ أَحَدࣱ ۝٢٦ يَٰٓأَيَّتُهَا
ٱلنَّفْسُ ٱلْمُطْمَئِنَّةُ ۝٢٧ ٱرْجِعِيٓ إِلَىٰ رَبِّكِ رَاضِيَةࣰ مَّرْضِيَّةࣰ ۝٢٨
فَٱدْخُلِي فِي عِبَٰدِي ۝٢٩ وَٱدْخُلِي جَنَّتِي ۝٣٠
\suraline{سُورَةُ البَلَدِ}
\bismline{بِسْمِ ٱللَّهِ ٱلرَّحْمَٰنِ ٱلرَّحِيمِ}
لَآ أُقْسِمُ بِهَٰذَا ٱلْبَلَدِ ۝١ وَأَنتَ حِلُّۢ بِهَٰذَا ٱلْبَلَدِ ۝٢ وَوَالِدࣲ وَمَا وَلَدَ ۝٣
لَقَدْ خَلَقْنَا ٱلْإِنسَٰنَ فِي كَبَدٍ ۝٤ أَيَحْسَبُ أَن لَّن يَقْدِرَ عَلَيْهِ
أَحَدࣱ ۝٥ يَقُولُ أَهْلَكْتُ مَالࣰا لُّبَدًا ۝٦ أَيَحْسَبُ أَن لَّمْ يَرَهُۥٓ أَحَدٌ ۝٧
أَلَمْ نَجْعَل لَّهُۥ عَيْنَيْنِ ۝٨ وَلِسَانࣰا وَشَفَتَيْنِ ۝٩ وَهَدَيْنَٰهُ
ٱلنَّجْدَيْنِ ۝١٠ فَلَا ٱقْتَحَمَ ٱلْعَقَبَةَ ۝١١ وَمَآ أَدْرَىٰكَ مَا ٱلْعَقَبَةُ ۝١٢
فَكُّ رَقَبَةٍ ۝١٣ أَوْ إِطْعَٰمࣱ فِي يَوْمࣲ ذِي مَسْغَبَةࣲ ۝١٤ يَتِيمࣰا ذَا مَقْرَبَةٍ ۝١٥
أَوْ مِسْكِينࣰا ذَا مَتْرَبَةࣲ ۝١٦ ثُمَّ كَانَ مِنَ ٱلَّذِينَ ءَامَنُوا۟ وَتَوَاصَوْا۟
بِٱلصَّبْرِ وَتَوَاصَوْا۟ بِٱلْمَرْحَمَةِ ۝١٧ أُو۟لَٰٓئِكَ أَصْحَٰبُ ٱلْمَيْمَنَةِ ۝١٨
وَٱلَّذِينَ كَفَرُوا۟ بِـَٔايَٰتِنَا هُمْ أَصْحَٰبُ ٱلْمَشْـَٔمَةِ ۝١٩ عَلَيْهِمْ نَارࣱ مُّؤْصَدَةُۢ ۝٢٠
\suraline{سُورَةُ الشَّمْسِ}
\bismline{بِسْمِ ٱللَّهِ ٱلرَّحْمَٰنِ ٱلرَّحِيمِ}
وَٱلشَّمْسِ وَضُحَىٰهَا ۝١ وَٱلْقَمَرِ إِذَا تَلَىٰهَا ۝٢ وَٱلنَّهَارِ إِذَا جَلَّىٰهَا ۝٣
وَٱلَّيْلِ إِذَا يَغْشَىٰهَا ۝٤ وَٱلسَّمَآءِ وَمَا بَنَىٰهَا ۝٥ وَٱلْأَرْضِ
وَمَا طَحَىٰهَا ۝٦ وَنَفْسࣲ وَمَا سَوَّىٰهَا ۝٧ فَأَلْهَمَهَا فُجُورَهَا
وَتَقْوَىٰهَا ۝٨ قَدْ أَفْلَحَ مَن زَكَّىٰهَا ۝٩ وَقَدْ خَابَ مَن دَسَّىٰهَا ۝١٠
كَذَّبَتْ ثَمُودُ بِطَغْوَىٰهَآ ۝١١ إِذِ ٱنۢبَعَثَ أَشْقَىٰهَا ۝١٢ فَقَالَ لَهُمْ
رَسُولُ ٱللَّهِ نَاقَةَ ٱللَّهِ وَسُقْيَٰهَا ۝١٣ فَكَذَّبُوهُ فَعَقَرُوهَا فَدَمْدَمَ
عَلَيْهِمْ رَبُّهُم بِذَنۢبِهِمْ فَسَوَّىٰهَا ۝١٤ وَلَا يَخَافُ عُقْبَٰهَا ۝١٥
\suraline{سُورَةُ اللَّيْلِ}
\bismline{بِسْمِ ٱللَّهِ ٱلرَّحْمَٰنِ ٱلرَّحِيمِ}
وَٱلَّيْلِ إِذَا يَغْشَىٰ ۝١ وَٱلنَّهَارِ إِذَا تَجَلَّىٰ ۝٢ وَمَا خَلَقَ ٱلذَّكَرَ وَٱلْأُنثَىٰٓ ۝٣
إِنَّ سَعْيَكُمْ لَشَتَّىٰ ۝٤ فَأَمَّا مَنْ أَعْطَىٰ وَٱتَّقَىٰ ۝٥ وَصَدَّقَ بِٱلْحُسْنَىٰ ۝٦
فَسَنُيَسِّرُهُۥ لِلْيُسْرَىٰ ۝٧ وَأَمَّا مَنۢ بَخِلَ وَٱسْتَغْنَىٰ ۝٨ وَكَذَّبَ بِٱلْحُسْنَىٰ ۝٩
فَسَنُيَسِّرُهُۥ لِلْعُسْرَىٰ ۝١٠ وَمَا يُغْنِي عَنْهُ مَالُهُۥٓ إِذَا تَرَدَّىٰٓ ۝١١ إِنَّ عَلَيْنَا
لَلْهُدَىٰ ۝١٢ وَإِنَّ لَنَا لَلْأٓخِرَةَ وَٱلْأُولَىٰ ۝١٣ فَأَنذَرْتُكُمْ نَارࣰا تَلَظَّىٰ ۝١٤
لَا يَصْلَىٰهَآ إِلَّا ٱلْأَشْقَى ۝١٥ ٱلَّذِي كَذَّبَ وَتَوَلَّىٰ ۝١٦ وَسَيُجَنَّبُهَا
ٱلْأَتْقَى ۝١٧ ٱلَّذِي يُؤْتِي مَالَهُۥ يَتَزَكَّىٰ ۝١٨ وَمَا لِأَحَدٍ عِندَهُۥ مِن نِّعْمَةࣲ
تُجْزَىٰٓ ۝١٩ إِلَّا ٱبْتِغَآءَ وَجْهِ رَبِّهِ ٱلْأَعْلَىٰ ۝٢٠ وَلَسَوْفَ يَرْضَىٰ ۝٢١
\suraline{سُورَةُ الضُّحَىٰ}
\bismline{بِسْمِ ٱللَّهِ ٱلرَّحْمَٰنِ ٱلرَّحِيمِ}
وَٱلضُّحَىٰ ۝١ وَٱلَّيْلِ إِذَا سَجَىٰ ۝٢ مَا وَدَّعَكَ رَبُّكَ وَمَا قَلَىٰ ۝٣
وَلَلْأٓخِرَةُ خَيْرࣱ لَّكَ مِنَ ٱلْأُولَىٰ ۝٤ وَلَسَوْفَ يُعْطِيكَ رَبُّكَ
فَتَرْضَىٰٓ ۝٥ أَلَمْ يَجِدْكَ يَتِيمࣰا فَـَٔاوَىٰ ۝٦ وَوَجَدَكَ ضَآلࣰّا فَهَدَىٰ ۝٧
وَوَجَدَكَ عَآئِلࣰا فَأَغْنَىٰ ۝٨ فَأَمَّا ٱلْيَتِيمَ فَلَا تَقْهَرْ ۝٩
وَأَمَّا ٱلسَّآئِلَ فَلَا تَنْهَرْ ۝١٠ وَأَمَّا بِنِعْمَةِ رَبِّكَ فَحَدِّثْ ۝١١
\suraline{سُورَةُ الشَّرْحِ}
\bismline{بِسْمِ ٱللَّهِ ٱلرَّحْمَٰنِ ٱلرَّحِيمِ}
أَلَمْ نَشْرَحْ لَكَ صَدْرَكَ ۝١ وَوَضَعْنَا عَنكَ وِزْرَكَ ۝٢
ٱلَّذِيٓ أَنقَضَ ظَهْرَكَ ۝٣ وَرَفَعْنَا لَكَ ذِكْرَكَ ۝٤ فَإِنَّ مَعَ ٱلْعُسْرِ يُسْرًا ۝٥
إِنَّ مَعَ ٱلْعُسْرِ يُسْرࣰا ۝٦ فَإِذَا فَرَغْتَ فَٱنصَبْ ۝٧ وَإِلَىٰ رَبِّكَ فَٱرْغَب ۝٨
\suraline{سُورَةُ التِّينِ}
\bismline{بِّسْمِ ٱللَّهِ ٱلرَّحْمَٰنِ ٱلرَّحِيمِ}
وَٱلتِّينِ وَٱلزَّيْتُونِ ۝١ وَطُورِ سِينِينَ ۝٢ وَهَٰذَا ٱلْبَلَدِ ٱلْأَمِينِ ۝٣
لَقَدْ خَلَقْنَا ٱلْإِنسَٰنَ فِيٓ أَحْسَنِ تَقْوِيمࣲ ۝٤ ثُمَّ رَدَدْنَٰهُ أَسْفَلَ سَٰفِلِينَ ۝٥
إِلَّا ٱلَّذِينَ ءَامَنُوا۟ وَعَمِلُوا۟ ٱلصَّٰلِحَٰتِ فَلَهُمْ أَجْرٌ غَيْرُ مَمْنُونࣲ ۝٦
فَمَا يُكَذِّبُكَ بَعْدُ بِٱلدِّينِ ۝٧ أَلَيْسَ ٱللَّهُ بِأَحْكَمِ ٱلْحَٰكِمِينَ ۝٨
\suraline{سُورَةُ العَلَقِ}
\bismline{بِسْمِ ٱللَّهِ ٱلرَّحْمَٰنِ ٱلرَّحِيمِ}
ٱقْرَأْ بِٱسْمِ رَبِّكَ ٱلَّذِي خَلَقَ ۝١ خَلَقَ ٱلْإِنسَٰنَ مِنْ عَلَقٍ ۝٢ ٱقْرَأْ
وَرَبُّكَ ٱلْأَكْرَمُ ۝٣ ٱلَّذِي عَلَّمَ بِٱلْقَلَمِ ۝٤ عَلَّمَ ٱلْإِنسَٰنَ
مَا لَمْ يَعْلَمْ ۝٥ كَلَّآ إِنَّ ٱلْإِنسَٰنَ لَيَطْغَىٰٓ ۝٦ أَن رَّءَاهُ ٱسْتَغْنَىٰٓ ۝٧
إِنَّ إِلَىٰ رَبِّكَ ٱلرُّجْعَىٰٓ ۝٨ أَرَءَيْتَ ٱلَّذِي يَنْهَىٰ ۝٩ عَبْدًا
إِذَا صَلَّىٰٓ ۝١٠ أَرَءَيْتَ إِن كَانَ عَلَى ٱلْهُدَىٰٓ ۝١١ أَوْ أَمَرَ بِٱلتَّقْوَىٰٓ ۝١٢
أَرَءَيْتَ إِن كَذَّبَ وَتَوَلَّىٰٓ ۝١٣ أَلَمْ يَعْلَم بِأَنَّ ٱللَّهَ يَرَىٰ ۝١٤ كَلَّا لَئِن لَّمْ يَنتَهِ
لَنَسْفَعَۢا بِٱلنَّاصِيَةِ ۝١٥ نَاصِيَةࣲ كَٰذِبَةٍ خَاطِئَةࣲ ۝١٦ فَلْيَدْعُ نَادِيَهُۥ ۝١٧
سَنَدْعُ ٱلزَّبَانِيَةَ ۝١٨ كَلَّا لَا تُطِعْهُ \sajdabar{وَٱسْجُدْ وَٱقْتَرِب}۩ ۝١٩
\suraline{سُورَةُ القَدْرِ}
\bismline{بِّسْمِ ٱللَّهِ ٱلرَّحْمَٰنِ ٱلرَّحِيمِ}
إِنَّآ أَنزَلْنَٰهُ فِي لَيْلَةِ ٱلْقَدْرِ ۝١ وَمَآ أَدْرَىٰكَ مَا لَيْلَةُ ٱلْقَدْرِ ۝٢
لَيْلَةُ ٱلْقَدْرِ خَيْرࣱ مِّنْ أَلْفِ شَهْرࣲ ۝٣ تَنَزَّلُ ٱلْمَلَٰٓئِكَةُ وَٱلرُّوحُ فِيهَا
بِإِذْنِ رَبِّهِم مِّن كُلِّ أَمْرࣲ ۝٤ سَلَٰمٌ هِيَ حَتَّىٰ مَطْلَعِ ٱلْفَجْرِ ۝٥
\suraline{سُورَةُ البَيِّنَةِ}
\bismline{بِسْمِ ٱللَّهِ ٱلرَّحْمَٰنِ ٱلرَّحِيمِ}
لَمْ يَكُنِ ٱلَّذِينَ كَفَرُوا۟ مِنْ أَهْلِ ٱلْكِتَٰبِ وَٱلْمُشْرِكِينَ مُنفَكِّينَ حَتَّىٰ
تَأْتِيَهُمُ ٱلْبَيِّنَةُ ۝١ رَسُولࣱ مِّنَ ٱللَّهِ يَتْلُوا۟ صُحُفࣰا مُّطَهَّرَةࣰ ۝٢ فِيهَا كُتُبࣱ
قَيِّمَةࣱ ۝٣ وَمَا تَفَرَّقَ ٱلَّذِينَ أُوتُوا۟ ٱلْكِتَٰبَ إِلَّا مِنۢ بَعْدِ مَا جَآءَتْهُمُ
ٱلْبَيِّنَةُ ۝٤ وَمَآ أُمِرُوٓا۟ إِلَّا لِيَعْبُدُوا۟ ٱللَّهَ مُخْلِصِينَ لَهُ ٱلدِّينَ
حُنَفَآءَ وَيُقِيمُوا۟ ٱلصَّلَوٰةَ وَيُؤْتُوا۟ ٱلزَّكَوٰةَۚ وَذَٰلِكَ دِينُ ٱلْقَيِّمَةِ ۝٥
إِنَّ ٱلَّذِينَ كَفَرُوا۟ مِنْ أَهْلِ ٱلْكِتَٰبِ وَٱلْمُشْرِكِينَ فِي نَارِ جَهَنَّمَ
خَٰلِدِينَ فِيهَآۚ أُو۟لَٰٓئِكَ هُمْ شَرُّ ٱلْبَرِيَّةِ ۝٦ إِنَّ ٱلَّذِينَ ءَامَنُوا۟
وَعَمِلُوا۟ ٱلصَّٰلِحَٰتِ أُو۟لَٰٓئِكَ هُمْ خَيْرُ ٱلْبَرِيَّةِ ۝٧ جَزَآؤُهُمْ
عِندَ رَبِّهِمْ جَنَّٰتُ عَدْنࣲ تَجْرِي مِن تَحْتِهَا ٱلْأَنْهَٰرُ خَٰلِدِينَ
فِيهَآ أَبَدࣰاۖ رَّضِيَ ٱللَّهُ عَنْهُمْ وَرَضُوا۟ عَنْهُۚ ذَٰلِكَ لِمَنْ خَشِيَ رَبَّهُۥ ۝٨
\suraline{سُورَةُ الزَّلْزَلَةِ}
\bismline{بِسْمِ ٱللَّهِ ٱلرَّحْمَٰنِ ٱلرَّحِيمِ}
إِذَا زُلْزِلَتِ ٱلْأَرْضُ زِلْزَالَهَا ۝١ وَأَخْرَجَتِ ٱلْأَرْضُ أَثْقَالَهَا ۝٢ وَقَالَ
ٱلْإِنسَٰنُ مَا لَهَا ۝٣ يَوْمَئِذࣲ تُحَدِّثُ أَخْبَارَهَا ۝٤ بِأَنَّ رَبَّكَ أَوْحَىٰ لَهَا ۝٥
يَوْمَئِذࣲ يَصْدُرُ ٱلنَّاسُ أَشْتَاتࣰا لِّيُرَوْا۟ أَعْمَٰلَهُمْ ۝٦ فَمَن يَعْمَلْ
مِثْقَالَ ذَرَّةٍ خَيْرࣰا يَرَهُۥ ۝٧ وَمَن يَعْمَلْ مِثْقَالَ ذَرَّةࣲ شَرࣰّا يَرَهُۥ ۝٨
\suraline{سُورَةُ العَادِيَاتِ}
\bismline{بِسْمِ ٱللَّهِ ٱلرَّحْمَٰنِ ٱلرَّحِيمِ}
وَٱلْعَٰدِيَٰتِ ضَبْحࣰا ۝١ فَٱلْمُورِيَٰتِ قَدْحࣰا ۝٢ فَٱلْمُغِيرَٰتِ
صُبْحࣰا ۝٣ فَأَثَرْنَ بِهِۦ نَقْعࣰا ۝٤ فَوَسَطْنَ بِهِۦ جَمْعًا ۝٥
إِنَّ ٱلْإِنسَٰنَ لِرَبِّهِۦ لَكَنُودࣱ ۝٦ وَإِنَّهُۥ عَلَىٰ ذَٰلِكَ لَشَهِيدࣱ ۝٧ وَإِنَّهُۥ لِحُبِّ
ٱلْخَيْرِ لَشَدِيدٌ ۝٨۞ أَفَلَا يَعْلَمُ إِذَا بُعْثِرَ مَا فِي ٱلْقُبُورِ ۝٩
وَحُصِّلَ مَا فِي ٱلصُّدُورِ ۝١٠ إِنَّ رَبَّهُم بِهِمْ يَوْمَئِذࣲ لَّخَبِيرُۢ ۝١١
\suraline{سُورَةُ القَارِعَةِ}
\bismline{بِسْمِ ٱللَّهِ ٱلرَّحْمَٰنِ ٱلرَّحِيمِ}
ٱلْقَارِعَةُ ۝١ مَا ٱلْقَارِعَةُ ۝٢ وَمَآ أَدْرَىٰكَ مَا ٱلْقَارِعَةُ ۝٣ يَوْمَ
يَكُونُ ٱلنَّاسُ كَٱلْفَرَاشِ ٱلْمَبْثُوثِ ۝٤ وَتَكُونُ ٱلْجِبَالُ كَٱلْعِهْنِ
ٱلْمَنفُوشِ ۝٥ فَأَمَّا مَن ثَقُلَتْ مَوَٰزِينُهُۥ ۝٦ فَهُوَ فِي عِيشَةࣲ
رَّاضِيَةࣲ ۝٧ وَأَمَّا مَنْ خَفَّتْ مَوَٰزِينُهُۥ ۝٨ فَأُمُّهُۥ هَاوِيَةࣱ ۝٩
وَمَآ أَدْرَىٰكَ مَا هِيَهْ ۝١٠ نَارٌ حَامِيَةُۢ ۝١١
\suraline{سُورَةُ التَّكَاثُرِ}
\bismline{بِسْمِ ٱللَّهِ ٱلرَّحْمَٰنِ ٱلرَّحِيمِ}
أَلْهَىٰكُمُ ٱلتَّكَاثُرُ ۝١ حَتَّىٰ زُرْتُمُ ٱلْمَقَابِرَ ۝٢ كَلَّا سَوْفَ تَعْلَمُونَ ۝٣ ثُمَّ
كَلَّا سَوْفَ تَعْلَمُونَ ۝٤ كَلَّا لَوْ تَعْلَمُونَ عِلْمَ ٱلْيَقِينِ ۝٥ لَتَرَوُنَّ ٱلْجَحِيمَ ۝٦
ثُمَّ لَتَرَوُنَّهَا عَيْنَ ٱلْيَقِينِ ۝٧ ثُمَّ لَتُسْـَٔلُنَّ يَوْمَئِذٍ عَنِ ٱلنَّعِيمِ ۝٨
%\looseness1
\newpage
\suraline{سُورَةُ العَصْرِ}
\bismline{بِسْمِ ٱللَّهِ ٱلرَّحْمَٰنِ ٱلرَّحِيمِ}
\centerline{\hbox to1\textwidth{وَٱلْعَصْرِ ۝١ إِنَّ ٱلْإِنسَٰنَ لَفِي خُسْرٍ ۝٢ إِلَّا ٱلَّذِينَ ءَامَنُوا۟}}
\centerline{\hbox to1\textwidth{وَعَمِلُوا۟ ٱلصَّٰلِحَٰتِ وَتَوَاصَوْا۟ بِٱلْحَقِّ وَتَوَاصَوْا۟ بِٱلصَّبْرِ ۝٣}}
\suraline{سُورَةُ الهُمَزَةِ}
\bismline{بِسْمِ ٱللَّهِ ٱلرَّحْمَٰنِ ٱلرَّحِيمِ}
\centerline{\hbox to1\textwidth{وَيْلࣱ لِّكُلِّ هُمَزَةࣲ لُّمَزَةٍ ۝١ ٱلَّذِي جَمَعَ مَالࣰا وَعَدَّدَهُۥ ۝٢}}
\centerline{\hbox to1\textwidth{يَحْسَبُ أَنَّ مَالَهُۥٓ أَخْلَدَهُۥ ۝٣ كَلَّاۖ لَيُنۢبَذَنَّ فِي ٱلْحُطَمَةِ ۝٤}}
\centerline{\hbox to1\textwidth{وَمَآ أَدْرَىٰكَ مَا ٱلْحُطَمَةُ ۝٥ نَارُ ٱللَّهِ ٱلْمُوقَدَةُ ۝٦ ٱلَّتِي تَطَّلِعُ}}
\centerline{\hbox to1\textwidth{عَلَى ٱلْأَفْـِٔدَةِ ۝٧ إِنَّهَا عَلَيْهِم مُّؤْصَدَةࣱ ۝٨ فِي عَمَدࣲ مُّمَدَّدَةِۭ ۝٩}}
\suraline{سُورَةُ الفِيلِ}
\bismline{بِسْمِ ٱللَّهِ ٱلرَّحْمَٰنِ ٱلرَّحِيمِ}
\centerline{\hbox to1\textwidth{أَلَمْ تَرَ كَيْفَ فَعَلَ رَبُّكَ بِأَصْحَٰبِ ٱلْفِيلِ ۝١ أَلَمْ يَجْعَلْ}}
\centerline{\hbox to1\textwidth{كَيْدَهُمْ فِي تَضْلِيلࣲ ۝٢ وَأَرْسَلَ عَلَيْهِمْ طَيْرًا أَبَابِيلَ ۝٣}}
\centerline{\hbox to1\textwidth{تَرْمِيهِم بِحِجَارَةࣲ مِّن سِجِّيلࣲ ۝٤ فَجَعَلَهُمْ كَعَصْفࣲ مَّأْكُولِۭ ۝٥}}
\newpage
\suraline{سُورَةُ قُرَيْشٍ}
\bismline{بِسْمِ ٱللَّهِ ٱلرَّحْمَٰنِ ٱلرَّحِيمِ}
لِإِيلَٰفِ قُرَيْشٍ ۝١ إِۦلَٰفِهِمْ رِحْلَةَ ٱلشِّتَآءِ وَٱلصَّيْفِ ۝٢
فَلْيَعْبُدُوا۟ رَبَّ هَٰذَا ٱلْبَيْتِ ۝٣ ٱلَّذِيٓ أَطْعَمَهُم
\centerline{\hbox to0.63\textwidth{مِّن جُوعࣲ وَءَامَنَهُم مِّنْ خَوْفِۭ ۝٤}}
\suraline{سُورَةُ المَاعُونِ}
\bismline{بِسْمِ ٱللَّهِ ٱلرَّحْمَٰنِ ٱلرَّحِيمِ}
أَرَءَيْتَ ٱلَّذِي يُكَذِّبُ بِٱلدِّينِ ۝١ فَذَٰلِكَ ٱلَّذِي يَدُعُّ
ٱلْيَتِيمَ ۝٢ وَلَا يَحُضُّ عَلَىٰ طَعَامِ ٱلْمِسْكِينِ ۝٣ فَوَيْلࣱ
لِّلْمُصَلِّينَ ۝٤ ٱلَّذِينَ هُمْ عَن صَلَاتِهِمْ سَاهُونَ ۝٥
\centerline{\hbox to0.9\textwidth{ٱلَّذِينَ هُمْ يُرَآءُونَ ۝٦ وَيَمْنَعُونَ ٱلْمَاعُونَ ۝٧}}
\suraline{سُورَةُ الكَوْثَرِ}
\bismline{بِسْمِ ٱللَّهِ ٱلرَّحْمَٰنِ ٱلرَّحِيمِ}
إِنَّآ أَعْطَيْنَٰكَ ٱلْكَوْثَرَ ۝١ فَصَلِّ لِرَبِّكَ وَٱنْحَرْ ۝٢
\centerline{\hbox to0.53\textwidth{إِنَّ شَانِئَكَ هُوَ ٱلْأَبْتَرُ ۝٣}}
\newpage
\suraline{سُورَةُ الكَافِرُونَ}
\bismline{بِسْمِ ٱللَّهِ ٱلرَّحْمَٰنِ ٱلرَّحِيمِ}
قُلْ يَٰٓأَيُّهَا ٱلْكَٰفِرُونَ ۝١ لَآ أَعْبُدُ مَا تَعْبُدُونَ ۝٢
وَلَآ أَنتُمْ عَٰبِدُونَ مَآ أَعْبُدُ ۝٣ وَلَآ أَنَا۠ عَابِدࣱ مَّا عَبَدتُّمْ ۝٤
وَلَآ أَنتُمْ عَٰبِدُونَ مَآ أَعْبُدُ ۝٥ لَكُمْ دِينُكُمْ وَلِيَ دِينِ ۝٦
\suraline{سُورَةُ النَّصْرِ}
\bismline{بِسْمِ ٱللَّهِ ٱلرَّحْمَٰنِ ٱلرَّحِيمِ}
إِذَا جَآءَ نَصْرُ ٱللَّهِ وَٱلْفَتْحُ ۝١ وَرَأَيْتَ ٱلنَّاسَ
يَدْخُلُونَ فِي دِينِ ٱللَّهِ أَفْوَاجࣰا ۝٢ فَسَبِّحْ بِحَمْدِ رَبِّكَ
\centerline{\hbox to0.66\textwidth{وَٱسْتَغْفِرْهُۚ إِنَّهُۥ كَانَ تَوَّابَۢا ۝٣}}
\suraline{سُورَةُ المَسَدِ}
\bismline{بِسْمِ ٱللَّهِ ٱلرَّحْمَٰنِ ٱلرَّحِيمِ}
\centerline{\hbox to1\textwidth{تَبَّتْ يَدَآ أَبِي لَهَبࣲ وَتَبَّ ۝١ مَآ أَغْنَىٰ عَنْهُ مَالُهُۥ وَمَا كَسَبَ ۝٢}}
سَيَصْلَىٰ نَارࣰا ذَاتَ لَهَبࣲ ۝٣ وَٱمْرَأَتُهُۥ حَمَّالَةَ ٱلْحَطَبِ ۝٤
\centerline{\hbox to0.60\textwidth{فِي جِيدِهَا حَبْلࣱ مِّن مَّسَدِۭ ۝٥}}
\newpage
\suraline{سُورَةُ الإِخْلَاصِ}
\bismline{بِسْمِ ٱللَّهِ ٱلرَّحْمَٰنِ ٱلرَّحِيمِ}
\centerline{\hbox to1\textwidth{قُلْ هُوَ ٱللَّهُ أَحَدٌ ۝١ ٱللَّهُ ٱلصَّمَدُ ۝٢ لَمْ يَلِدْ وَلَمْ يُولَدْ ۝٣}}
\centerline{\hbox to0.55\textwidth{وَلَمْ يَكُن لَّهُۥ كُفُوًا أَحَدُۢ ۝٤}}
\suraline{سُورَةُ الفَلَقِ}
\bismline{بِسْمِ ٱللَّهِ ٱلرَّحْمَٰنِ ٱلرَّحِيمِ}
\centerline{\hbox to1\textwidth{قُلْ أَعُوذُ بِرَبِّ ٱلْفَلَقِ ۝١ مِن شَرِّ مَا خَلَقَ ۝٢ وَمِن شَرِّ}}
\centerline{\hbox to1\textwidth{غَاسِقٍ إِذَا وَقَبَ ۝٣ وَمِن شَرِّ ٱلنَّفَّٰثَٰتِ فِي ٱلْعُقَدِ ۝٤}}
\centerline{\hbox to0.55\textwidth{وَمِن شَرِّ حَاسِدٍ إِذَا حَسَدَ ۝٥}}
\suraline{سُورَةُ النَّاسِ}
\bismline{بِسْمِ ٱللَّهِ ٱلرَّحْمَٰنِ ٱلرَّحِيمِ}
\centerline{\hbox to1\textwidth{قُلْ أَعُوذُ بِرَبِّ ٱلنَّاسِ ۝١ مَلِكِ ٱلنَّاسِ ۝٢ إِلَٰهِ}}
\centerline{\hbox to1\textwidth{ٱلنَّاسِ ۝٣ مِن شَرِّ ٱلْوَسْوَاسِ ٱلْخَنَّاسِ ۝٤ ٱلَّذِي}}
\centerline{\hbox to0.675\textwidth{يُوَسْوِسُ فِي صُدُورِ ٱلنَّاسِ ۝٥}}
\centerline{\hbox to0.5\textwidth{مِنَ ٱلْجِنَّةِ وَٱلنَّاسِ ۝٦}}
