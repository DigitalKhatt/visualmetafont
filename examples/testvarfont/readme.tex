%\directlua{require('mobdebug').start()}
\documentclass{article}
\usepackage[letterpaper]{geometry}
\usepackage{graphicx}
\usepackage{bookmark}
\usepackage{digitalkhatt}
\usepackage{luatextra}
\usepackage{setspace}


\newcommand{\DigitalKhatt}{DigitalKhatt}
\newcommand{\LuaOTFLoad}{\mbox{\texttt{luaotfload}}}

\title{DigitalKhatt \LuaLaTeX{} Package}
\date{December 2022}
\author{Amine Anane}
\begin{document}
\maketitle
This \LuaLaTeX{} package enables to use the \DigitalKhatt{} variable font in the \TeX{} system. In the previous version, I have to patch \LuaTeX{} and \LuaOTFLoad{} to use the font.
In the current version, a new \LuaOTFLoad{} plugin mode, named \mbox{\texttt{digitalkhatt}}, has been defined, which uses its own files as follows.
\begin{itemize}
  \item {\bf digitalkhatt.sty}: The \LaTeX{} package style file
  \item {\bf digitalkhatt.lua}: Main file
  \item {\bf digitalkhatt-define.lua}: copied from \mbox{\texttt{luaotfload-harf-define.lua}} and adapted to \DigitalKhatt{}
  \item {\bf digitalkhatt-plug.lua}: copied from \mbox{\texttt{luaotfload-harf-plug.lua}} and adapted to \DigitalKhatt{}
  \item {\bf digitalkhatt-cff2.otf}: DigitalKhatt OpenType variable font extended with new lookup types (such as finite state machine lookup) and justification table
  \item {\bf digitalkhatt\_glyphnames.lua and digitalkhatt\_lookupnames.lua}: accompaniments to {\bf digitalkhatt-cff2.otf}. Needed for lookup callbacks.
    Currently they are used to extend the fatha mark above some stretched glyphs.
  \item {\bf lpdf-ini-digitalkhatt.lua}: Copy of the Context file {\bf lpdf-ini.lua} Mainly  used to define PDF Type3 fonts.
  \item {\bf charstrings.lua}: Parsing of charstring by glyph id.
  \item {\bf fontloader-digitalkhatt-cff2.lua}: Copied from {\bf fontloader-font-cff.lua} and patched to allow only the current extended glyph to be parsed and not the entire file. Used with {\bf charstrings.lua}.
  \item {\bf luadigitalkhatt.[dll,so]}: The native lua module : HarfBuzz + Justification Algorithm + new OpenType subtables 
\end{itemize}
To use the package, all the above files has to be copied in the working directory and in the main tex file you add the command \verb!\usepackage{digitalkhatt}! in the preamble.
To use the font you call the command \verb!\digitalkhatt{fontsize}! giving the font size as parameter. For example, here the word welcome using a font size of 12pt {\digitalkhatt{12pt}مرحبًا}.
Since the package uses a Lua native module, you should call \LuaLaTeX{} using the \verb!--shell-escape! command line argument as follows.
\begin{verbatim}
lualatex --shell-escape readme.tex
\end{verbatim}
The justification is disabled by default. Here an example of a paragraph using 18pt font size without justification.\par
{\digitalkhatt{18pt}\selectfont\pardir TRT
الٓمٓ ۝١ ذَٰلِكَ ٱلْكِتَٰبُ لَا رَيْبَۛ فِيهِۛ هُدࣰى
لِّلْمُتَّقِينَ ۝٢ ٱلَّذِينَ يُؤْمِنُونَ بِٱلْغَيْبِ وَيُقِيمُونَ ٱلصَّلَوٰةَ
وَمِمَّا رَزَقْنَٰهُمْ يُنفِقُونَ ۝٣ وَٱلَّذِينَ يُؤْمِنُونَ بِمَآ أُنزِلَ
إِلَيْكَ وَمَآ أُنزِلَ مِن قَبْلِكَ وَبِٱلْأٓخِرَةِ هُمْ يُوقِنُونَ ۝٤
أُو۟لَٰٓئِكَ عَلَىٰ هُدࣰى مِّن رَّبِّهِمْۖ وَأُو۟لَٰٓئِكَ
هُمُ ٱلْمُفْلِحُونَ ۝٥
\par}
You can enable and disable justification using the commands \verb!\enableDKJust! and \verb!\disableDKJust!. Below is the same paragraph using justification.
Some alternates are now applied in the first line. Also by using justification the third line is compressed by shrinking the kashida between joining letters.\par
{\digitalkhatt{18pt}\enableDKJust\pardir TRT
الٓمٓ ۝١ ذَٰلِكَ ٱلْكِتَٰبُ لَا رَيْبَۛ فِيهِۛ هُدࣰى
لِّلْمُتَّقِينَ ۝٢ ٱلَّذِينَ يُؤْمِنُونَ بِٱلْغَيْبِ وَيُقِيمُونَ ٱلصَّلَوٰةَ
وَمِمَّا رَزَقْنَٰهُمْ يُنفِقُونَ ۝٣ وَٱلَّذِينَ يُؤْمِنُونَ بِمَآ أُنزِلَ
إِلَيْكَ وَمَآ أُنزِلَ مِن قَبْلِكَ وَبِٱلْأٓخِرَةِ هُمْ يُوقِنُونَ ۝٤
أُو۟لَٰٓئِكَ عَلَىٰ هُدࣰى مِّن رَّبِّهِمْۖ وَأُو۟لَٰٓئِكَ
هُمُ ٱلْمُفْلِحُونَ ۝٥
\par}

For now, the justification only applies to lines or paragraphs which are only composed of the same \DigitalKhatt{} font without changing direction.

When there is an overfull or underfull box, you can decrease or increase the font size to limit the overfull or underfull amount.
The command \verb!\setDKMinExpFactor! sets the minimum ratio (i.e. desired line width / current line width) allowed to decrease the font size and the command \verb!\setDKMaxExpFactor! sets the maximum ratio allowed to increase the font size.
By default the minimum expansion ration is set to 0.98 and the maximum expansion ratio is set to 1 (i.e. There is no font expansion when the line is loose).
To use this feature you must first enable font expansion with the command \verb!\enableDKFontExpansion!.
For instance, the following example does not use justification 
\def\hline{\hbox to\textwidth}
\vskip10pt
\vbox{
\bodydir TRT
\digitalkhatt{18pt}
\hline{\hbox to0.75\textwidth{سَنُقْرِئُكَ فَلَا تَنسَىٰٓ ۝٦ إِلَّا مَا شَآءَ ٱللَّهُۚ إِنَّهُۥ يَعْلَمُ ٱلْجَهْرَ وَمَا يَخْفَىٰ ۝٧}\hfill}
\hline{\hbox to0.75\textwidth{وَنُيَسِّرُكَ لِلْيُسْرَىٰ ۝٨ فَذَكِّرْ إِن نَّفَعَتِ ٱلذِّكْرَىٰ ۝٩ سَيَذَّكَّرُ مَن يَخْشَىٰ ۝١٠}\hfill}
\hline{\hbox to0.75\textwidth{وَيَتَجَنَّبُهَا ٱلْأَشْقَى ۝١١ ٱلَّذِي يَصْلَى ٱلنَّارَ ٱلْكُبْرَىٰ ۝١٢ ثُمَّ لَا يَمُوتُ}\hfill}
}\vskip10pt
Below is the same example using justification.
\vskip10pt
\vbox{
\bodydir TRT
\digitalkhatt{18pt}\enableDKJust
\hline{\hbox to0.75\textwidth{سَنُقْرِئُكَ فَلَا تَنسَىٰٓ ۝٦ إِلَّا مَا شَآءَ ٱللَّهُۚ إِنَّهُۥ يَعْلَمُ ٱلْجَهْرَ وَمَا يَخْفَىٰ ۝٧}\hfill}
\hline{\hbox to0.75\textwidth{وَنُيَسِّرُكَ لِلْيُسْرَىٰ ۝٨ فَذَكِّرْ إِن نَّفَعَتِ ٱلذِّكْرَىٰ ۝٩ سَيَذَّكَّرُ مَن يَخْشَىٰ ۝١٠}\hfill}
\hline{\hbox to0.75\textwidth{وَيَتَجَنَّبُهَا ٱلْأَشْقَى ۝١١ ٱلَّذِي يَصْلَى ٱلنَّارَ ٱلْكُبْرَىٰ ۝١٢ ثُمَّ لَا يَمُوتُ}\hfill}
}\vskip10pt
Below is the same example using justification and enabling font expansion by calling the commands \verb!\enableDKFontExpansion! and \verb!\setDKMinExpFactor{0.95}!. 
Notice that the font size of the second line has decreased to fit the width of the line. 
\vskip10pt
\vbox{
\bodydir TRT\enableDKFontExpansion\setDKMinExpFactor{0.95}
\digitalkhatt{18pt}\enableDKJust
\hline{\hbox to0.75\textwidth{سَنُقْرِئُكَ فَلَا تَنسَىٰٓ ۝٦ إِلَّا مَا شَآءَ ٱللَّهُۚ إِنَّهُۥ يَعْلَمُ ٱلْجَهْرَ وَمَا يَخْفَىٰ ۝٧}\hfill}
\hline{\hbox to0.75\textwidth{وَنُيَسِّرُكَ لِلْيُسْرَىٰ ۝٨ فَذَكِّرْ إِن نَّفَعَتِ ٱلذِّكْرَىٰ ۝٩ سَيَذَّكَّرُ مَن يَخْشَىٰ ۝١٠}\hfill}
\hline{\hbox to0.75\textwidth{وَيَتَجَنَّبُهَا ٱلْأَشْقَى ۝١١ ٱلَّذِي يَصْلَى ٱلنَّارَ ٱلْكُبْرَىٰ ۝١٢ ثُمَّ لَا يَمُوتُ}\hfill}
}\vskip10pt

There are also two commands \verb!\enableDKTajweedColor! and \verb!\disableDKTajweedColor! to enable and disable Tajweed colors.\par
\vskip10pt
\vbox{\bodydir TRT\enableDKTajweedColor\enableDKJust\digitalkhatt{18pt}%
\centerline{\hbox to0.8\textwidth{قُلْ هُوَ ٱللَّهُ أَحَدٌ ۝١ ٱللَّهُ ٱلصَّمَدُ ۝٢ لَمْ يَلِدْ وَلَمْ يُولَدْ ۝٣}}
\centerline{\hbox to0.44\textwidth{وَلَمْ يَكُن لَّهُۥ كُفُوًا أَحَدُۢ ۝٤}}
}
\vskip10pt

The \DigitalKhatt{} project is still work-in progress. If you encounter a problem, please file an issue on GitHub.


\end{document} 